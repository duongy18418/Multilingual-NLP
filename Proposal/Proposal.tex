\documentclass[sigconf]{acmart}
\pagestyle{fancy}
\usepackage{mdwlist}
\usepackage{amsmath}
\fancyhead{}
\settopmatter{printacmref=false, printfolios=false}
\setcopyright{none}
\settopmatter{printacmref=false}

\begin{document}
\title{Natural Language Processing in Marketing Strategy for Chinese-Speaking Countries}
\author{Kitty Duong}
\affiliation{
  \institution{University of Windsor}
}
\email{duongy@uwindsor.ca}

\begin{abstract}
Besides English, Chinese is listed as one of the most spoken languages in the world, with 21 countries considering it to be the mother tongue by part of their population. Due to its popularity, Chinese also has many varieties across multiple countries and regions. When an algorithm is developed for translation and multilingual usage, the developers will sometimes choose only one specific variant of Chinese to implement. However, given the growing popularity of global marketing strategy, companies need to adjust their marketing material multiple times when entering new markets, even though some of these markets use Chinese. Given the needs, we propose conducting a research project for a new algorithm using Natural Language Processing to make translating and adjusting marketing materials to multiple variants of Chinese more efficient.
\end{abstract}

\keywords{Chinese translation NLP, multilingual NLP, NLP}

\maketitle

\section{Introduction}
With the growing and expanding of global marketing strategies, companies are looking for the most efficient way to utilize their resources when entering a new foreign market. To adjust the marketing material to tailor to a certain foreign market, companies will need to understand their target audiences, their culture, and most importantly, their language. However, given the diversity of some languages, even though some countries use the same language, they might use different written variants or dialects. 

\section{Motivation}
In this research proposal, we propose to consider the importance of 

\subsection{Motivating Example}

\section{Problem Definition}
Given a company based in the United States with a marketing strategy $\mathcal{M}$ for a product that has been successfully implemented in the US market, and a set of all written variants of the Chinese language $\mathcal{L}$. We present a global marketing strategy $g$, where $g \subset \mathcal{M}$, to implement for a set of countries $\mathcal{C}$ that use a variant of Chinese $v$, where $v \in \mathcal{L}$, as one of their languages. Our main objective is for a country $c \in \mathcal{C}$, to accurately determine $v$ to translate the content of $g$. For instance, Taiwan uses a variant of Chinese called Mandarin, we want to set the target language $v$ to Mandarin and translate $g$ using proper language structure and grammar.

\section{Team Justification}
Kitty Duong: Work on Abstract, and problem definition.
Miaomiao Zhang: Work on introduction and motivation.
\end{document}
